%%%%%%%%%%%%%%%%%%%%%%%%%%%%%%%%%%%%%%%%%
%
% CMPT 435
% Some Semester
% Assignment 1
%
%%%%%%%%%%%%%%%%%%%%%%%%%%%%%%%%%%%%%%%%%

%%%%%%%%%%%%%%%%%%%%%%%%%%%%%%%%%%%%%%%%%
% Short Sectioned Assignment
% LaTeX Template
% Version 1.0 (5/5/12)
%
% This template has been downloaded from: http://www.LaTeXTemplates.com
% Original author: % Frits Wenneker (http://www.howtotex.com)
% License: CC BY-NC-SA 3.0 (http://creativecommons.org/licenses/by-nc-sa/3.0/)
% Modified by Alan G. Labouseur  - alan@labouseur.com
%
%%%%%%%%%%%%%%%%%%%%%%%%%%%%%%%%%%%%%%%%%

%----------------------------------------------------------------------------------------
%	PACKAGES AND OTHER DOCUMENT CONFIGURATIONS
%----------------------------------------------------------------------------------------

\documentclass[letterpaper, 10pt,DIV=13]{scrartcl} 
\usepackage{minted}

\usepackage[T1]{fontenc} % Use 8-bit encoding that has 256 glyphs
\usepackage[english]{babel} % English language/hyphenation
\usepackage{amsmath,amsfonts,amsthm,xfrac} % Math packages
\usepackage{sectsty} % Allows customizing section commands
\usepackage{graphicx}
\usepackage[lined,linesnumbered,commentsnumbered]{algorithm2e}
\usepackage{listings}
\usepackage{parskip}
\usepackage{lastpage}

\allsectionsfont{\normalfont\scshape} % Make all section titles in default font and small caps.

\usepackage{fancyhdr} % Custom headers and footers
\pagestyle{fancyplain} % Makes all pages in the document conform to the custom headers and footers

\fancyhead{} % No page header - if you want one, create it in the same way as the footers below
\fancyfoot[L]{} % Empty left footer
\fancyfoot[C]{} % Empty center footer
\fancyfoot[R]{page \thepage\ of \pageref{LastPage}} % Page numbering for right footer

\renewcommand{\headrulewidth}{0pt} % Remove header underlines
\renewcommand{\footrulewidth}{0pt} % Remove footer underlines
\setlength{\headheight}{13.6pt} % Customize the height of the header

\numberwithin{equation}{section} % Number equations within sections (i.e. 1.1, 1.2, 2.1, 2.2 instead of 1, 2, 3, 4)
\numberwithin{figure}{section} % Number figures within sections (i.e. 1.1, 1.2, 2.1, 2.2 instead of 1, 2, 3, 4)
\numberwithin{table}{section} % Number tables within sections (i.e. 1.1, 1.2, 2.1, 2.2 instead of 1, 2, 3, 4)

\setlength\parindent{0pt} % Removes all indentation from paragraphs.

\binoppenalty=3000
\relpenalty=3000

%----------------------------------------------------------------------------------------
%	TITLE SECTION
%----------------------------------------------------------------------------------------

\newcommand{\horrule}[1]{\rule{\linewidth}{#1}} % Create horizontal rule command with 1 argument of height

\title{	
   \normalfont \normalsize 
   \textsc{CMPT 435 - Fall 2022 - Dr. Labouseur} \\[10pt] % Header stuff.
   \horrule{0.5pt} \\[0.25cm] 	% Top horizontal rule
   \huge Assignment One  \\     	    % Assignment title
   \horrule{0.5pt} \\[0.25cm] 	% Bottom horizontal rule
}

\author{Shannon Brady \\ \normalsize shannon.brady2@Marist.edu}

\date{\normalsize\today} 	% Today's date.

\begin{document}
\maketitle % Print the title


%----------------------------------------------------------------------------------------
%   start Node and Linked List
%----------------------------------------------------------------------------------------
\section{Node and LinkedList}

\subsection{Node Class}
A Node has the following attributes:
\begin{enumerate}
  \item Name (a string value)
  \item Next (a Node value indicating the next item in the linked list, stack, or queue)
\end{enumerate}

\begin{minted}
[
frame=lines,
framesep=2mm,
baselinestretch=1.2,
fontsize=\footnotesize,
linenos
]
{java}
class Node {

    private String name = "";
    private Node next = null;

    /* Node class constructor */
    public Node(String name) {
        this.name = name;
        this.next = null;
    } // Node

    /* Getters and Setters */
    public String getName() {
        return this.name;
    } // getName

    public Node getNext() {
        return this.next;
    } // getNext

    public void setName(String newName) {
        this.name = newName;
    } // setName

    public void setNext(Node newNext) {
        this.next = newNext;
    } // setNext

    @Override
    public String toString() {
        return("Name: " + this.getName() + "\n");
    } // toString
} // Node
\end{minted}

\subsection{LinkedList Class}
The LinkedList class implements a linked list data structure by utilizing the Node class. A LinkedList has the following attributes:
\begin{enumerate}
    \item head (a Node value indicating the first item in the linked list)
    \item length (an integer value indicating list length)
\end{enumerate}

A LinkedList  has the following methods: 
\begin{enumerate}
    \item add<string> (add a new Node, line 13) 
        \begin{enumerate}
            \item If the head is null, the new node becomes the head; otherwise, add the node to the end of the list.
        \end{enumerate}
    \item remove<string> (remove an existing Node, line 33)
        \begin{enumerate}
            \item First we check if the target string matches a node name in the list. Next, if there's not a previous node, set the head to the current node's next. If there is a previous node, connect the previous node's next to the current node's next.
        \end{enumerate}
    \item print (print all Nodes in the list, line 58)
    \item isEmpty (checks if list is empty, line 74)
        \begin{enumerate}
            \item If the head is null, then the list is empty.
        \end{enumerate}
\end{enumerate}

\begin{minted}
[
frame=lines,
framesep=2mm,
baselinestretch=1.2,
fontsize=\footnotesize,
linenos
]
{java}

class LinkedList {

    private Node head = null;
    private int length = 0;

    /* LinkedList class constructor */
    LinkedList() {
        this.head = null;
        this.length = 0;
    } // LinkedList

    public void add(String str) {
        Node newNode = new Node(str);

        if (this.head == null) {
            // If the list is empty, set the head
            this.head = newNode;
        } else {
            // Otherwise traverse entire list and add newNode to the end
            Node lastNode = this.head;
            while (lastNode.getNext() != null) {
                lastNode = lastNode.getNext();
            }

            // lastNode.next = newNode;
            lastNode.setNext(newNode);
        }
        //increment length
        this.length++;
    } // add

    public void remove(String str) {
        Node currNode = this.head;
        Node prevNode = null;
        int i =0;
        while (currNode != null) {
            if (str.equals(currNode.getName())) {
                if (prevNode == null) {
                    // Removing the head of the list
                    // Update head node
                    this.head = currNode.getNext();
                } else {
                    // Connect prevNode's next to currNode's next
                    prevNode.setNext(currNode.getNext());
                    // CurrNode is removed
                    currNode.setNext(null);
                }
            }
            i++;
            prevNode = currNode;
            currNode = currNode.getNext();
        }
        //decrement length
        this.length--;
    } // remove

    public void print() {
        // Prints all nodes in the list
        Node currNode = this.head;
        int i = 0;

        while (currNode != null) {
            System.out.println("Index: "+ i + "\n" +
                                currNode.toString());

            // Set the new next
            currNode = currNode.getNext();
            i++;
        }
        System.out.println();
    } // print

    public boolean isEmpty() {
        if (this.head == null) {
            return true;
        } else {
            return false;
        }
    } // isEmpty

    /* Getters and Setters */
    public LinkedList getLinkedList() {
        return this;
    } // getLinkedList

    public Node getHead() {
        return this.head;
    } // getHead

    public void setHead(Node newHead) {
        this.head = newHead;
    } // setHead

}
\end{minted}
%----------------------------------------------------------------------------------------
%   end Node and Linked List
%----------------------------------------------------------------------------------------


%----------------------------------------------------------------------------------------
%   start Stack
%----------------------------------------------------------------------------------------
\section{Stack Class}
The Stack class implements a stack data structure by utilizing the Node class. A Stack has the following attributes:
\begin{enumerate}
    \item top (a Node value indicating the first item in the stack)
    \item length (an integer value indicating stack length)
\end{enumerate}

A Stack has the following methods: 
\begin{enumerate}
    \item pop (remove a Node from the stack, line 11) 
        \begin{enumerate}
            \item If top is not null, then set the top to the next item in the stack.
        \end{enumerate}
    \item push (add a Node to the stack, line 22)
        \begin{enumerate}
            \item The new Node becomes the top of the stack.
        \end{enumerate}
    \item print (print all Nodes in the stack, line 32)
    \item isEmpty (checks if stack is empty, line 44)
        \begin{enumerate}
            \item If the top is null, then the stack is empty.
        \end{enumerate}
\end{enumerate}

\begin{minted}
[
frame=lines,
framesep=2mm,
baselinestretch=1.2,
fontsize=\footnotesize,
linenos
]
{java}
class Stack {

    private Node top = null;
    private int length = 0;

    /* Stack class constructor */
    Stack() {
        this.length = 0;
    } //Stack

    public Node pop() {
        Node x = null;
        if (top != null) {
            x = this.top;
            this.top = x.getNext();

            this.length--;
        }
        return x;
    } // pop

    public void push(String str) {
        Node newNode = new Node(str);

        // newNode.next = this.top;
        newNode.setNext(this.top);
        this.top = newNode;

        this.length++;
    } // push

    public void print() {
        // Prints all nodes in the stack
        Node currNode = this.top;

        while (currNode != null) {
            System.out.println(currNode.toString());

            // Set the new next
            currNode = currNode.getNext();
        }
    } // print

    public boolean isEmpty() {
        if (this.top == null) {
            return true;
        } else return false;
    } // isEmpty

    /* Getters and Setters */
    public Stack getStack() {
        return this;
    } // getStack

    public Node getTop() {
        return this.top;
    } // getTop

    public int getLength() {
        return this.length;
    } // getLength

    public void setTop(Node newTop) {
        this.top = newTop;
    } // setTop
}

\end{minted}

%----------------------------------------------------------------------------------------
%   end Stack
%----------------------------------------------------------------------------------------

%----------------------------------------------------------------------------------------
%   start Queue
%----------------------------------------------------------------------------------------
\section{Queue Class}
The Queue class implements a queue data structure by utilizing the Node class. A Queue has the following attributes:
\begin{enumerate}
    \item head (a Node value indicating the first item in the queue)
    \item tail (a Node value indicating the last item in the queue)
    \item length (an integer value indicating the length of the queue)
\end{enumerate}

A Queue has the following methods: 
\begin{enumerate}
    \item enqueue (add a Node to the end of the queue, line 14) 
        \begin{enumerate}
            \item If the queue is empty, then the head and tail both point to the same Node; otherwise only the tail is updated.
        \end{enumerate}
    \item dequeue (remove a Node from the front of the Queue, line 33)
        \begin{enumerate}
            \item If the queue is not empty, update the head to the next Node
        \end{enumerate}
    \item print (print all Nodes in the Queue, line 47)
    \item isEmpty (checks if queue is empty, line 63)
        \begin{enumerate}
            \item If the head is null, then the queue is empty.
        \end{enumerate}
\end{enumerate}

\begin{minted}
[
frame=lines,
framesep=2mm,
baselinestretch=1.2,
fontsize=\footnotesize,
linenos
]
{java}
class Queue {

    private Node head = null;
    private Node tail = null;
    private int length = 0;

    /* Queue class constructor */
    Queue() {
        this.head = null;
        this.tail = null;
        this.length = 0;
    }

    public void enqueue(String str) {
        Node newNode = new Node(str);

        if (this.head == null) {
            // Head and tail are the same if queue length is 1
            this.head = newNode;
            this.tail = newNode;
        } else {
            // Make the tail's next the new node
            this.tail.setNext(newNode);
            // Make the new node the tail
            this.tail = newNode;
            // Set the new node next value to null
            newNode.setNext(null);
        }

        this.length++;
    } // enqueue

    public Node dequeue() {
        Node x = null;
        if (this.head != null) {
            // If the queue isn't empty, grab the current head node
            x = this.head;
            // Update the head
            this.head = x.getNext();
            x.setNext(null);
            // Decrement the queue length
            this.length--;
        }
        return x;
    } // dequeue

    public void print() {
        // Prints all nodes in the list
        Node currNode = this.head;
        int i = 0;

        while (currNode != null) {
            System.out.println("Index: "+ i + "\n" +
                                currNode.toString());

            // Set the new next
            currNode = currNode.getNext();
            i++;
        }
        System.out.println();
    }

    public boolean isEmpty() {
        if (this.head == null) {
            return true;
        } else {
            return false;
        }
    } // isEmpty

    /* Getters and Setters */
    public Node getHead() {
        return this.head;
    } // getHead

    public Node getTail() {
        return this.tail;
    } // getTail

    public int getLength() {
        return this.length;
    } // getLength

    public void setHead(Node newHead) {
        this.head = newHead;
    } // setHead

    public void setTail(Node newTail) {
        this.tail = newTail;
    } // setTail
}
\end{minted}

%----------------------------------------------------------------------------------------
%   end Queue
%----------------------------------------------------------------------------------------

%----------------------------------------------------------------------------------------
%   start Main and Test
%----------------------------------------------------------------------------------------
\section{Test and Main}
\subsection{Test Class}
The Test class uses various methods to test each of the previously outlined data structures.

A Test has the following methods: 
\begin{enumerate}
    \item testMyLinkedList (validates add, remove, and isEmpty methods, line 16)
    \item testMyStack (validates pop, push, and isEmpty methods, line 49)
    \item testMyQueue (validates enqueue, dequeue, and isEmpty methods, line 94)
    \item checkPalindrome (identifies all palindromes in a text file, line 153)
        \begin{enumerate}
            \item Read an input file and add all lines of text to an array.
            \item Loop through all items in the array and remove all whitespace and special symbols, then make all letters lowercase.
            \item For each line of text, push each character to a stack and a queue. Then loop through either the stack or the queue (does not matter since length is equivalent) and pop and dequeue a letter from the stack and queue, respectively. 
            \item If the letters returned do not match at any given point, then the phrase cannot be a palindrome.
            \item Add all palindromes to a linked list and print the results.
        \end{enumerate}
\end{enumerate}
\begin{minted}
[
frame=lines,
framesep=2mm,
baselinestretch=1.2,
fontsize=\footnotesize,
linenos
]
{java}
import java.io.File;
import java.io.FileNotFoundException;
import java.util.Scanner;

class Test {

    Test() {
    }

    /* Testing Methods:
     *      testMyLinkedList()
     *      testMyStack()
     *      testMyQueue()
     *      checkPalindrome()
     */
    public void testMyLinkedList() {
        // Create a linked list of really awesome songs
        LinkedList songs = new LinkedList();

        // Validate the list is empty
        if (songs.isEmpty()) {
            System.out.println("Your list is empty.");
        }

        // Add some really awesome songs
        songs.add("Particles - Nothing But Thieves");
        songs.add("Sugarcoat - Kid Bloom");
        songs.add("Little One - Highly Suspect");
        songs.add("Why Are Sundays So Depressing - The Strokes");
        songs.add("My Honest Face - Inhaler");
        songs.add("Safari Song - Greta Van Fleet");
        songs.add("Down the Road - Dirty Honey");

        // Print the list of really awesome songs
        System.out.println("Awesome Songs");
        System.out.println("---------");
        songs.print();

        // Make sure remove is removing
        songs.remove("My Honest Face - Inhaler");
        songs.remove("Particles - Nothing But Thieves");
        songs.remove("Down the Road - Dirty Honey");

        System.out.println("Updated Awesome Songs");
        System.out.println("---------");
        songs.print();
    }

    public void testMyStack() {
        // Create a stack of words that rhyme with stack
        Stack rhymes = new Stack();

        // Validate stack is empty
        if (rhymes.isEmpty()) {
            System.out.println("Your stack is empty.");
        }

        // Add some rhymes
        rhymes.push("Rack");
        rhymes.push("Slack");
        rhymes.push("Whack");
        rhymes.push("Snack");
        rhymes.push("Quack");
        rhymes.push("Hack");
        rhymes.push(("Soundtrack"));

        // Print da rhymes
        System.out.println("Rhymes 1");
        System.out.println("---------");
        rhymes.print();

        // Make sure pop() is poppin...
        System.out.println("\nPop() Test:");
        System.out.println("---------");

        Stack rhymes_2 = new Stack();
        int length = rhymes.getLength();

        for (int i=0; i<length; i++) {
            // Pop and print
            Node x = rhymes.pop();
            System.out.println(x.toString());

            // Push to new stack, order is now flipped
            rhymes_2.push(x.getName());
        }

        // Print rhymes again, in reverse order
        System.out.println("\nRhymes 2");
        System.out.println("---------");
        rhymes_2.print();
    }

    public void testMyQueue() {
        // Create a queue of people waiting to checkout at Stop & Shop
        Queue checkout = new Queue();

        // Add the peeps
        checkout.enqueue("Karen");
        checkout.enqueue("Evan");
        checkout.enqueue("David");
        checkout.enqueue("Marco");
        checkout.enqueue("Shannon");
        checkout.enqueue("Dan");
        checkout.enqueue("Alan");

        System.out.println("Checkout:");
        System.out.println("---------");
        checkout.print();

        // Karen is now being helped
        Node karen = checkout.dequeue();
        System.out.println("Checkout:");
        System.out.println("---------");
        checkout.print();

        // Karen, in Karen fashion, made a scene asking for a manager...
        Queue customerService = new Queue();
        customerService.enqueue(karen.getName());
        System.out.println("Customer Service:");
        System.out.println("---------");
        customerService.print();

        // Alan is now next in line
        int length = checkout.getLength();
        for (int i=0; i<length-1; i++) {
            Node customer = checkout.dequeue();
        }
        System.out.println("Checkout:");
        System.out.println("---------");
        checkout.print();

        // Karen was told to get back on the checkout line
        // She tried to cut Alan but he didn't fall for her tricks
        checkout.enqueue(karen.getName());
        System.out.println("Checkout:");
        System.out.println("---------");
        checkout.print();

        // Finally they all go home
        length = checkout.getLength();
        for (int i=0; i<length; i++) {
            Node customer = checkout.dequeue();
        }

        System.out.println("Checkout:");
        System.out.println("---------");
        if (checkout.isEmpty()) {
            System.out.println("The checkout line is empty.");
        }
    }

    public void checkPalindrome() {
        // magicitems.txt
        String[] items = new String[666];

        // palindromes.txt
        // String[] items = new String[13];

        LinkedList palindromes = new LinkedList();
        Stack stack = new Stack();
        Queue queue = new Queue();
        int i = 0;
        
        try {
            File file = new File("magicitems.txt");
            // File file = new File("palindromes.txt");
            Scanner myReader = new Scanner(file);

            while (myReader.hasNextLine()) {
            String data = myReader.nextLine();
            items[i] = data;
            i++;
            }
            myReader.close();
        } catch (FileNotFoundException e) {
            System.out.println("An error occurred.");
            e.printStackTrace();
        }

        for (int w=0; w<items.length; w++) {
            // Normally I'd initialize this to false, but it seems to make more sense to look for all the words that aren't palindromes.
            // A non-palindrome and palindrome can both have matching characters on opposing ends of the string.
            // A palindrome will never have mismatching characters, a non-palindrome will.
            boolean isPalindrome = true;

            // Remove all whitespace and symbols
            items[w] = items[w].replaceAll("[\\s,\\.'\\(\\)+-]", "");

            // Convert to all lower case
            items[w] = items[w].toLowerCase();

            for (int j=0; j<items[w].length(); j++) {
                // Loop through all the character in the string 
                // and add each character to a stack and queue, respectively
                stack.push(String.valueOf(items[w].charAt(j)));
                queue.enqueue(String.valueOf(items[w].charAt(j)));
            }

            int length = stack.getLength();
            for (int c=0; c<length; c++) {
                // Pop and dequeue each letter
                String char1 = stack.pop().getName();
                String char2 = queue.dequeue().getName();

                if (!(char1.equals(char2))) {
                    // A word cannot be a palindrome if both chars are not the same
                    isPalindrome = false;
                }
            }
            if (isPalindrome) {
                // Add all palindromes to a separate list
                palindromes.add(items[w]);
            }
        }
        palindromes.print();
    }
}
\end{minted}

\subsection{Main Class}
The Main class creates an instance of the Test class and calls each testing method.
\begin{minted}
[
frame=lines,
framesep=2mm,
baselinestretch=1.2,
fontsize=\footnotesize,
linenos
]
{java}
class Assignment_1 {
    public static void main(String[] args) {
        
        Test test = new Test();

        // Test each of the data structures
        test.testMyLinkedList();
        test.testMyStack();
        test.testMyQueue();

        // Check for palindromes
        test.checkPalindrome();
    }
}
\end{minted}
%----------------------------------------------------------------------------------------
%   end Main and Test
%----------------------------------------------------------------------------------------

\end{document}
